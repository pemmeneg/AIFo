\section{Classifier Evaluation}
\subsection{Confusion Matrix}
\includegraphics[width=\linewidth]{./img/confusion_matrix.png}
\textbf{Mean Accuracy:}
\begin{itemize}
    \item How often is the classifier correct?
    \item $A = (t_p + t_n) / n$
\end{itemize}
\textbf{Mean Error:}
\begin{itemize}
    \item How often is the classifier wrong?
    \item $E = (f_p + f_n) / n$
\end{itemize}
\textbf{Precision:}
\begin{itemize}
    \item When the prediction is 1, how often is it correct?
    \item $P = t_p / (t_p + f_p)$
\end{itemize}
\textbf{Sensitivity, Recall, True Positive Rate (TPR):}
\begin{itemize}
    \item How often the prediction is 1 when it's actually 1
    \item $R = t_p / (t_p + f_n)$
\end{itemize}
\textbf{Miss Rate, False Negative Rate (FNR)}
\begin{itemize}
    \item $MR = 1 - TPR$
\end{itemize}
\textbf{False Positive Rate (FPR)}
\begin{itemize}
    \item $f_p / (f_p + t_n)$
\end{itemize}
\textbf{True Negative Rate (TNR)}
\begin{itemize}
    \item $t_n / (t_n + f_p)$
    \item $1 - FPR$
\end{itemize}

\subsection{Why Accuracy is not enough?}
\begin{itemize}
    \item If the prediction is constant the accuracy may still look decent
    \item E.g. allways predict false
    \item 90\% of the data is false
    \item Accuracy = 90\% (decent)
    \item Precision = 0
    \item Recall = 0
\end{itemize}

\subsection{Precision vs. Recall}
\begin{itemize}
    \item Increasing precision reduces Recall and vice versa
    \item Threshold is a business decision (depending on goals)
\end{itemize}

\subsection{Receiver Operating Characteristics}
\begin{itemize}
    \item Multiple FPR / TPR needed to draw a curve
    \item Für eine Kurve müsste man den Classifier mit allen Schwellenwerten im Interval auswerten
    \item Defined by FPR and TPR as x and y axes
    \item Visualizes tradeoff between TP (benefits) and FP (cost)
\end{itemize}
\begin{center}
    \includegraphics[width=0.4\linewidth]{./img/roc.png}
\end{center}
\textbf{Area under the curve}
\begin{itemize}
    \item Area under the ROC curve
    \item Shows how well the TPR and FPR is looking in the aggregate
    \item The greater the area under the curve, the higher the quality of the model
    \item The greater the area, the higher the ratio of TP to FP
\end{itemize}

\section{KNN}
\subsection{Linear Seperability}
\includegraphics[width=1\linewidth]{./img/linear_sep.png}

\begin{itemize}
    \item Based on logistic regression model, you can draw a line
    \item This is the Linear decision boundary
    \item If a simple line perfectly seperates the classes, then the classes are said to be linearely seperable
\end{itemize}

\subsection{Non-Linear decision boundary}
\begin{itemize}
    \item When classes are not linearly seperable
    \item Resort to polynomial terms
\end{itemize}

\subsection{k-Neares Neighbors (KNN)}
\begin{itemize}
    \item A datapoint is know by the company it keeps
    \item Computes $k$ nearest neighbours
    \item Returns the most frequent class of the $k$ neighbours
\end{itemize}
\includegraphics[width=0.8\linewidth]{./img/knn.png}
\subsubsection{Distance Metric}
\textbf{Cosine Distance (Skalar)}
\[ \frac{x_{1}*x_{2}}{||x_{1}||*||x_{2}||} \]
\textbf{Manhattan Disance (Quadrat um 90deg gedreht)}
\[ \sum_{i=1}^{n} |x_{1,n} - x_{2,n}|\] 
\textbf{Euclidean Distance (Kreis)}
\[ \sqrt{\sum_{i=1}^n (x_{i} - x_{i})^2} \]
\textbf{Minkowski Distance}
\subsubsection{Advantages}
\begin{itemize}
    \item Easy and simple ML model
    \item Few hyperparameters to tune
\end{itemize}

\subsubsection{Disadvantages}
\begin{itemize}
    \item $k$ should be wisely selected
    \item Large computation cost during runtime if sample size is large
    \item Not efficient for high dimensional datasets
    \item Proper scaling should be provided for fair treatment among features
\end{itemize}

\subsubsection{Hyperparameters}
\begin{itemize}
    \item \textbf{K Value}: how many neighbours to participate in the KNN algo.
    \item \textbf{Distance Function}: Euclidean distance is most used
\end{itemize}